\section{Putting it all together}
The learning curve for using the underlying SQL and linked data technologies
itself presents an obstacle to publishing and using log data, but most use
cases for publishers, annotators and analysts are based on only a few queries.
To mitigate this obstacle we provide tools that guide users through common
tasks without requiring direct use of SQL or SPARQL. This does not preclude
direct use of the underlying technology by those comfortable with it.

Finally, we present case studies on systems at different sites illustrating the mechanism,
benefit, and general applicability of the tools and schema presented here. Cases include:

\paragraph{User investigation of system events impacting job performance} Application
performance can be impacted by events which are apparent in log data but not exposed to 
users or not apparent to someone without appropriate domain knowledge to interpret the
log data. Our framework enables investigation by users or support staff by exposing 
expert translations of significant occurrences without necessarily exposing logs beyond
their normal security and privacy domain. In addition, the framework enables users to
identify and specify to support staff relevant slices of logs they cannot directly 
access, reducing the effort required by support staff to aid in such questions.

In this example, we show how a user can investigate underlying reasons for and
occurrences of performance variation issues by integrated access of expert annotations 
of CPU throttling events, link degrades, and memory errors which occur in disparate 
Cray logs in conjunction with historic job data.

\paragraph{Administrator event investigation}
The historical state of the system includes not only events captured by the logs, but
human-invoked actions as well, such as DIMM replacement, software upgrades, and cable reseating.
Our annotation system enables labeling and search of both human-defined and system-defined events
in a consistent way, and the return of such events in a format suitable for plotting or additional analysis.

This example shows how the investigation of component faults and resolution can be facilitated by our
framework. Sequences of failures followed by manual resolution action are easily discoverable and visualized
in a timeline. The easy determination of the time between problem onset and resolution is useful for
reporting of the system impact of particular component problems. Finally, long term analysis is used to
detect components (e.g., nodes) which indicate more faults over time, regardless of replacement, which can indicate that
the root cause of problems is environmental (e.g., temperature) rather than component-based.

A related example demonstrates characterization of the impacts of
events which are resolved by the system itself: network failure events which should be resolved by the automatic recalculation
of network routes. Using our framework on production and
fault injection data, we show how the timescales of impact are easily discovered, as well as the discovery
of events for which successful rerouting did not occur. Identification of these occurrences also then facilitates
direction to regions of the logs for more in-depth analysis as to the circumstances of the failure.







