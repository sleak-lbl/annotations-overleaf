\subsection{RDF Vocabulary}

The need for a decentralized mechanism for publishing information whose 
\emph{meaning} is machine-readable has been known for more than two decades
\textcolor{red}{cite https://www.w3.org/Talks/WWW94Tim/}. Since then a 
collection of tools and technologies supporting this has accumulated, with
perhaps the most significant being the specification of RDF as a data 
interchange format for the World Wide Web. \textcolor{red}{cite https://www.w3.org/RDF/?}
In RDF \emph{things} (concepts or concrete items) are represented as URIs
and arranged in \emph{triples} of a \emph{subject}, a \emph{predicate} and 
an \emph{object}. 

% do this as a figure, see Ann's example
\begin{figure*}
\begin{minted}{turtle}
@prefix nersc: <http://portal.nersc.gov/project/mpccc/sleak/nersc#> .
@prefix rdfs: <http://www.w3.org/2000/01/rdf-schema#> .
@prefix foaf: <http://xmlns.com/foaf/0.1/> .
nersc:nersc rdfs:type foaf:Organization .
\end{minted}

\caption{A triple of (subject, predicate, object) describes an edge 
in an RDF graph. The \texttt{Turtle}\textcolor{red}{cite} syntax shown
here aids human readability by condensing URIs into a prefix and a suffix,
so \texttt{rdfs:type} expands as
\texttt{<http://www.w3.org/2000/01/rdf-schema\#type>}.}
\label{f:rdftriples}
\end{figure*}

For example, in Figure~\ref{f:rdftriples} the subject is a URI that 
we've chosen to associate with NERSC. In reality, this is a web address where we can 
(TODO make this more readable)
the URI up to the final ``/'' is a web address the authors
can publish files from, where we have added an RDF file called ``nersc.ttl'' 
which includes an entry also called ``nersc''. If you point a web browser 
there you will see a ``404'' error because the ``.ttl'' suffix is missing 
(other public namespaces solve this with HTTP content negotiation) - but the 
point is, we are confident that this is a unique URI we can use as an 
identifier for NERSC. (And if a better identifier for NERSC comes along 
we can link the two, also with RDF).

The predicate here is one predefined by the RDF Schema to indicate the ``type''
of a thing.

And the object is a URI predefined by the Friend-of-a-friend ontology to
indicate an Organization.

A collection of triples forms a graph whose edges have well-defined meanings, 
allowing information about nodes to be inferred from their relationships to 
other nodes.

Crucially, an RDF graph can be queried using SPARQL \textcolor{red}{cite}, a
graph-query language with a look-and-feel based on SQL. A SPARQL query 
arranges variables into a set of triples and returns sets of URIs for which
the triples for a true statement. For example, the following SPARQL query
will return the name and the web page for each node whose type is 
a subclass of \texttt{foaf:Agent}. \texttt{foaf:Agent} is a superclass for a Person, a 
Group or an Organization, so this query in English is ``list the name and 
webpage for each Person, Group or Organization in this graph''. (The 
\texttt{rdfs:subClassOf*} syntax indicates that the query should follow 
\texttt{rdfs:subClassOf} edges to any depth until a \texttt{foaf:Agent} is encountered)

\begin{minted}{sparql}
SELECT ?name ?page 
WHERE {
    ?uri a ?type .
    ?type rdfs:subClassOf* foaf:Agent .
    ?uri foaf:name ?name .
    ?uri foaf:page ?page .
}
\end{minted}

\textcolor{red}{TODO: draw this as a diagram, probably explains it better}
\textcolor{red}{TODO: prob need a ref to an "intro to semantic web" resource}

\subsubsection{Why RDF?}

-- next: why is RDF useful for solving our problem

\subsubsection{The vocabulary}

\textcolor{red}{TODO include a diagram}


-- some key things to call out:
\begin{itemize}
\item someone publishing a logset doesn't need to know many people to get their 
      logset into the graph - a curator of their local (site) catalog is enough. 
      That curator then knows curators of at least one other catalog and thus 
      gets the local 
      subgraph into the global graph
\item someone using the graph doesn't need to know who published what - they can
      query the graph itself and get metadata about what is out there, including contact 
      information for data they don't directly have access to. This allows them to 
      solve specific access limitations in a locally-appropriate way
\end{itemize}

 
