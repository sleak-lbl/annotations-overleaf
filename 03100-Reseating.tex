\subsection{HSN Congestion}
\label{s:congestion}
%%%%%%%%%%%%%%%%%%%%%%%%%

In order to investigate network conditions, we chose to search for annotations involving the
word ''congest''. Query results are given in Figure~\ref{f:congest}. This looks like the
expected output of response to a congestion conditions, though it was surprising
on a system of this size, with candidate nodes and applications
as computed by the system.

\begin{figure*}
\begin{annol}
# query for annotations where any of the text fields (e.g., description, LDcategory) contain the word 'congest'
python get.py  -t congest -f table annotations
id      authorid   starttime              endtime                description                                                      logfiles    LDcategory    components    balerpatternid
756163    acg      2015-04-28 10:15:44    2015-04-28 10:15:44    System computing and listing congestion candidate applications   nlrd    NE    ["unknown"]    704
756168    acg      2015-04-28 10:15:44    2015-04-28 10:15:44    System computing and listing congestion candidate nodes          nlrd    NE    ["unknown"]    705
...
756167    acg      2015-04-28 10:32:06    2015-04-28 10:32:06    System computing and listing congestion candidate applications   nlrd    NE    ["unknown"]    704
756172    acg      2015-04-28 10:32:06    2015-04-28 10:32:06    System computing and listing congestion candidate nodes          nlrd    NE    ["unknown"]    705
\end{annol}
\caption{Congestion response annotations occur 5 times within 15 minutes. The annotation regarding
candidate applications drove investigation of the \texttt{nlrd} log file, but no applications
were listed.}
\label{f:congest}
\end{figure*}

These annotations guided us to the \texttt{nlrd} in search of applications of interest. 
The link from annotation to relevant slice of the source logfile is a planned capability
of the graph-query tool, however that functionality was not yet available so a visual 
inspection of the relevant file was used.

There were, however, \emph{no} applications listed.
As a result, we then queried for all annotations around this time window to find indication
of a non-application congestion cause.

A query for annotations between 10:00 and 10:32 on that day resulted in 300 annotations, however there are only 7 distinct ones.
The reduction in log lines to annotation instances makes investigation of time ranges tractable and eases discovery
of similar event instances. Other than the ones in Figure~\ref{f:congest}, the rest dealt with problems with a single component \texttt{c0-0c1s8a0n0} and system
response to congestion.
\begin{itemize}
\item  192 occurrences for component \texttt{c0-0c1s8a0n0} of \scriptsize\texttt{Correctable memory error.  This may result in degraded performance.}\normalsize
\item  47 occurrences for component \texttt{c0-0c1s8a0n0} of \scriptsize\texttt{Component failed.}\normalsize
\item \scriptsize\texttt{Telling all blades to throttle network bandwidth. This should result in decreased network injection.}\normalsize
\item \scriptsize\texttt{Telling all blades to unthrottle network bandwidth. This should enable increased network injection.}\normalsize
\item \scriptsize\texttt{Unthrottling the service blades only}\normalsize
\end{itemize}


While we cannot be sure from the annotations alone that failure in this component was the cause of the congestion, it
is clearly a strong suspect.

Querying for annotations for \texttt{c0-0c1s8a0n0} revealed that the component's problems of these 2 types
started on 2015-04-02 10:22:47 and ended on 2015-04-30 at 07:16:52.
Narrowing down the root cause of the problems is difficult, however, because of a number of deliberately induced failures
during facilities testing which occurred that day. While we have put in a manual annotation for \scriptsize\texttt{Facilities testing}\normalsize,
the distributed system which we envision would have enabled the Facilities staff to annotate in more detail the exact testing which occurred.
Currently the \scriptsize\texttt{Facilities testing}\normalsize annotation has to serve as a indicator to examine the logs in which
indications of induced fan and power failures occurred.

The resolution of the problem is discovered by using a depth search to query annotations of related components: here \texttt{-d 2}
includes two levels of parents (\texttt{c0-0c1s8a0} and \texttt{c0-0c1s8}) children (none), and any unknown/supremum components.
The depth was chosen with the expectation that resolution would occur due to actions at the Aries or blade level.

While roughly 500 annotations occurred in response to the query,
only about 30 distinct annotations occurred. This is in contrast to the raw log files in which over 153,000 log lines occurred.
The annotations make it easy to understand the sequence of events. Extracted annotations are in Figure~\ref{f:congestresolve}.
First an annotation of a system administrator, 'abc', action, generically assigned to the day (green) confirms that the blade is being reseated in response
to errors. Warmswaps of the blade occur (cyan), however, while the warmswaps report as successful, timeouts waiting for items in the Outstanding Request Buffer (ORB)
result in the ORB being 'scrubbed', delaying the recovery (red). The annotations help with the understanding of the ORB scrub related events.
Eventually the blade is added back (green) successfully and the blade is then booted.

\begin{figure*}
\begin{annol}

# query for annotations between the time frame of interest for the named component and any components within a depth of 2
python get.py  -c c0-0c1s8a0n0 -d 2 -s "2015-04-30 07:00:00" -e "2015-04-30 10:00:00" -f table annotations

id	authorid	starttime	endtime		endstate    description	logfiles	LDcategory	components	balerpatternid

<@\textbf{\textcolor{green}{4	abc	2015-04-30 00:00:01	2015-04-30 23:59:59	aries errors	blade reseated	Blade reseating in response to aries errors	NE	["c0-0c1s8"]}}@>
<@\textbf{\textcolor{cyan}{865013	acg	2015-04-30 07:10:14	2015-04-30 07:10:15	1	xtwarmswap remove	commands	NO	["c0-0c1s8"]}}@>
<@\textbf{\textcolor{cyan}{865797	acg	2015-04-30 07:15:50	2015-04-30 07:15:50	1	xtwarmswap remove	commands	NO	["c0-0c1s8"]}}@>
<@\textbf{\textcolor{cyan}{866043	acg	2015-04-30 07:16:47	2015-04-30 07:16:47	1	xtwarmswap remove	commands	NO	["c0-0c1s8"]}}@>
<@\textbf{\textcolor{cyan}{865976	acg	2015-04-30 07:16:55	2015-04-30 07:17:25	0	xtwarmswap remove	commands	NO	["unknown"]}}@>
766569	acg	2015-04-30 07:16:56	2015-04-30 07:16:56		Handling Warm swap for partition.	nlrd	NO	["unknown"]	455
866491	acg	2015-04-30 07:16:56	2015-04-30 07:16:56	0	xtcli set_alert		commands	NE	["c0-0c1s8a0"]
752251	acg	2015-04-30 07:17:03	2015-04-30 07:17:03		Setting alerts due to failures. A network reroute is required	nlrd	NE	["unknown"]	498
756229	acg	2015-04-30 07:17:10	2015-04-30 07:17:10		Quiescing the network. This should result in decreased network injection.	nlrd	NE	["unknown"]	509
756239	acg	2015-04-30 07:17:10	2015-04-30 07:17:10		Finished quiescing the network. 	nlrd	NE	["unknown"]	512
865961	acg	2015-04-30 07:17:16	2015-04-30 07:17:23	0	xtcli set_alert		commands	NO	["unknown"]
756132	acg	2015-04-30 07:17:25	2015-04-30 07:17:25		Telling all blades to unthrottle network bandwidth. This should enable increased network injection.	nlrd	NE	["unknown"]	423
756249	acg	2015-04-30 07:17:25	2015-04-30 07:17:25		Unquiescing the network. This will allow normal traffic injection to resume.	nlrd	NE	["unknown"]	554
756259	acg	2015-04-30 07:17:25	2015-04-30 07:17:25		Finished unquiescing the network. This will allow normal traffic injection to resume.	nlrd	NE	["unknown"]	557
<@\textbf{\textcolor{green}{766581	acg	2015-04-30 07:17:25	2015-04-30 07:17:25		Warm swap was successful. This is in response to a operation intended to reset/reinit/replace a component (including network components).	nlrd	NO	["unknown"]	566}}@>
766593	acg	2015-04-30 07:17:25	2015-04-30 07:17:25		The recovery operation for a failed link(s) was successful	nlrd	NE	["unknown"]	563
766603	acg	2015-04-30 07:17:25	2015-04-30 07:17:25		Done handling warm swap. This may not necessarily indicate success (?). This is in response to a operation intended to reset/reinit/replace a component (including network components).		nlrd	NO	["unknown"]	561
757133	acg	2015-04-30 07:17:42	2015-04-30 07:17:42		Starting to quiesce the node (node id might be in nodemask).		controllermessages	NE	["c0-0c1s8"]	2661
758693	acg	2015-04-30 07:17:42	2015-04-30 07:17:42		Finished quiescing the node.		controllermessages	NE	["c0-0c1s8"]	2666
<@\textbf{\textcolor{red}{763170	acg	2015-04-30 07:17:42	2015-04-30 07:17:42		Starting ORB scrub -- removing items in the Outstanding Request Buffer since its been too long for those messages	controllermessages	NE	["c0-0c1s8"]	2660}}@>
764343	acg	2015-04-30 07:17:52	2015-04-30 07:17:52		Finishing ORB scrub -- done removing items in the Outstanding Request Buffer since its been too long for those messages	controllermessages	NE	["c0-0c1s8"]	2669
760262	acg	2015-04-30 07:17:53	2015-04-30 07:17:53		Starting to unquiesce the node.		controllermessages	NE	["c0-0c1s8"]	2670
761831	acg	2015-04-30 07:17:53	2015-04-30 07:17:53		Finished unquiescing the node.		controllermessages	NE	["c0-0c1s8"]	2676
<@\textbf{\textcolor{red}{764926	acg	2015-04-30 07:17:53	2015-04-30 07:17:53		ORB timeout on node (nodes are in the message)		nlrd	NE	["unknown"]	435}}@>
<@\textbf{\textcolor{red}{757134	acg	2015-04-30 07:17:54	2015-04-30 07:17:54		Starting to quiesce the node (node id might be in nodemask).		controllermessages	NE	["c0-0c1s8"]	2661}}@>
758694	acg	2015-04-30 07:17:54	2015-04-30 07:17:54		Finished quiescing the node.		controllermessages	NE	["c0-0c1s8"]	2666
<@\textbf{\textcolor{red}{763171	acg	2015-04-30 07:17:54	2015-04-30 07:17:54		Starting ORB scrub -- removing items in the Outstanding Request Buffer since its been too long for those messages	controllermessages	NE	["c0-0c1s8"]	2660}}@>
764344	acg	2015-04-30 07:18:04	2015-04-30 07:18:04		Finishing ORB scrub -- done removing items in the Outstanding Request Buffer since its been too long for those messages	controllermessages	NE	["c0-0c1s8"]	2669
760263	acg	2015-04-30 07:18:05	2015-04-30 07:18:05		Starting to unquiesce the node.		controllermessages	NE	["c0-0c1s8"]	2670
761832	acg	2015-04-30 07:18:05	2015-04-30 07:18:05		Finished unquiescing the node.		controllermessages	NE	["c0-0c1s8"]	2676
<@\textbf{\textcolor{red}{764927	acg	2015-04-30 07:18:05	2015-04-30 07:18:05		ORB timeout on node (nodes are in the message)		nlrd	NE	["unknown"]	435}}@>

.....

<@\textbf{\textcolor{green}{866031	acg	2015-04-30 07:56:36	2015-04-30 08:03:27	0	xtwarmswap add	1	commands	NO	["c0-0c1s8"]}}@>
766571	acg	2015-04-30 07:56:39	2015-04-30 07:56:39		Handling Warm swap for partition.	nlrd	NO	["unknown"]	455
865720	acg	2015-04-30 07:56:39	2015-04-30 07:56:40	0	xtcli clr_alert	1	commands	NO	["c0-0c1s8"]
865819	acg	2015-04-30 07:56:39	2015-04-30 07:56:39	0	xtcli clr_alert	1	commands	NE	["c0-0c1s8a0"]
866784	acg	2015-04-30 07:56:40	2015-04-30 07:56:40	0	xtcli clr_warn	1	commands	NO	["c0-0c1s8"]
865012	acg	2015-04-30 07:56:46	2015-04-30 08:02:21	0	xtcli clr_warn	1	commands	NO	["c0-0c1s8"]
752253	acg	2015-04-30 08:02:22	2015-04-30 08:02:22		Setting alerts due to failures. A network reroute is required	nlrd	NE	["unknown"]	498
756231	acg	2015-04-30 08:03:08	2015-04-30 08:03:08		Quiescing the network. This should result in decreased network injection.	nlrd	NE	["unknown"]	509
756241	acg	2015-04-30 08:03:08	2015-04-30 08:03:08		Finished quiescing the network. 	nlrd	NE	["unknown"]	512
756251	acg	2015-04-30 08:03:23	2015-04-30 08:03:23		Unquiescing the network. This will allow normal traffic injection to resume.	nlrd	NE	["unknown"]	554
756261	acg	2015-04-30 08:03:23	2015-04-30 08:03:23		Finished unquiescing the network. This will allow normal traffic injection to resume.	nlrd	NE	["unknown"]	557
756134	acg	2015-04-30 08:03:24	2015-04-30 08:03:24		Telling all blades to unthrottle network bandwidth. This should enable increased network injection.	nlrd	NE	["unknown"]	423
<@\textbf{\textcolor{green}{766583	acg	2015-04-30 08:03:24	2015-04-30 08:03:24		Warm swap was successful. This is in response to a operation intended to reset/reinit/replace a component (including network components).	nlrd	NO	["unknown"]	566}}@>
766595	acg	2015-04-30 08:03:24	2015-04-30 08:03:24		The recovery operation for a failed link(s) was successful	nlrd	NE	["unknown"]	563
766605	acg	2015-04-30 08:03:24	2015-04-30 08:03:24		Done handling warm swap. This may not necessarily indicate success (?). This is in response to a operation intended to reset/reinit/replace a component (including network components).		nlrd	NO	["unknown"]	561
<@\textbf{\textcolor{green}{865000	acg	2015-04-30 08:08:34	2015-04-30 08:08:39	0	xtcli boot	1	commands	NO	["c0-0c1s8"]}}@>
\end{annol}
\caption{A blade reseating was performed to resolve blade problems which led to the congestion event.
Multiple iterations of scrubbing the Outstanding Request Buffer (ORB) were needed which
delayed resolution. The annotation of the system administrator (identified by 'abc') action supports the diagnosis.
}
\label{f:congestresolve}
\end{figure*}


The annotations additionally make it easy to compare and investigate timescales of similar events. For instance,
a recovery analysis might be based on the occurrences and durations of the warmswap sequences. This case
might appear of longer duration that others, and the intervening ORB scrubbing events that
needed to be handled for full recovery would be easily apparent.

