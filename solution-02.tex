\subsection{Annotation Schema}

In this subsection we describe the major fields necessary to support the
architectural requirements that pertain to the annotations themselves
(as opposed to the remote discovery infrastructure). The prototype
implementation is an SQLite database, so we descirbe in in terms
of that implemention, however that is not required. More generally
this would be descirbed in terms of accessor APIs independent of
the implementation beneath them.

The annotations table is defined as:
\begin{small}
\begin{minted}{sql}
CREATE TABLE 'annotations' (
       'id' integer,
       'authorid' char(3) NOT NULL,
       'starttime' datetime NOT NULL,
       'endtime' datetime NOT NULL,
       'startstate' text,
       'endstate' text,
       'description' text NOT NULL,
       'manual' boolean,
       'logfiles' text,
       'system' text,
       'systemdown' boolean,
       'LDcatgroup' text \
          REFERENCES 'LDgroups'('id'),
       'LDcategory' text,
       'LDtag' text,
       'components' text,
        balerpatternid integer,
        PRIMARY KEY('id','authorid')
        );
\end{minted}
\end{small}

The major fields for an annotations include \texttt{description}, \texttt{starttime}, \texttt{end time},
\texttt{system}, and \texttt{components} involved.
The description is a descriptive translation of a log messge or event (Examples
are given in Section~\ref{s:annotations}); it is this description to which we
mainly refer when we refer to an annotation.
The \texttt{authorid} identifies the author of an annotation, who
is responsible for the determination of the content of the assignable fields
(such as the description); the authorid and annotation id together form a unique key.
Time and component fields refer to a specific instance of a log line or event
occurence. Thus multiple annotated log line instances may share the same description,
but have different time and component values.

Other information is used to help with context and search.
For example, \texttt{startstate} and \texttt{endstate}, which
are subjective and can be used to record things like a component
that starts a faulty but is repaired by the action of the annotations,
for example, if the event is recording that a DIMM was changed.
\texttt{Systemdown} is meant to aid in the search for events that result
in full system failure.

\texttt{Manual} is used to indicate manually performed
events, such as an administrator action to take down a node as opposed
to the system taking down a node because it failed a health check.
Knowledge of this can be used to more accurately determine the
number of true failure events and for assessing the effectiveness
and availability of system resilience mechansims vs required
human intervention.

A set of fields, \texttt{LDXXX} are available to give contextual inforatmion.
At the moment only \texttt{LDcategory} is used; this is used to define the major
subsystem relevant to the annotation. The current options are: \texttt{node/blade},
\texttt{scheduler}, \texttt{storage}, \texttt{network}, \texttt{cooling/facilities/sensors},
\texttt{power}, \texttt{system software}, \texttt{datawarp}, and \texttt{unknown}.

Note that event may have many relationships. We choose to limit them to one each to address
the requirement of searchability. It is expected that exploration will facilitate dsicovery
of events, even when the first order assocation may be off; also discovery of assocationed
events in different subsytems can help understanding of how events propagate in a system.

Annotations may also refer to other annotations; this is called a \texttt{metaannotation}.
An example might be where an annotation refers to a particular facilities
test and a meta annotations refers to the entire set of tests.

\texttt{Components} can include the compute and any supporting subsystems,
such as storage and facilities related elements. For the Cray system
itself, we define \texttt{node}, \texttt{blade}, \texttt{chassis},
\texttt{cabinet}, \texttt{router}, \texttt{tile}, \texttt{link}, \texttt{nic},
\texttt{smw}, and \texttt{other}.

In order to enable dsicovery of events which are either reported on related
compoentns or that propagate amont components we support the defintion
of \texttt{architectures}. For the Cray system itself, we define three.

The \texttt{physical} architecture consists of parent-child or
container-contained assocations, such as a cabinet is the parent of
3 chassis. For the network components, the router is a child of the blade.
he router is a parent of the links and nics. The physical architecture
can be populated from system defintation (e.g., XE or XC with the
correct number of cabinets for a system).

In order to support the fact that components may be identified by differnet
names in different log messages, an \texttt{alias} table defines those conventions.
This is largely cname, nid, and IP addresses determined from \texttt{/etc/hosts}
or the output for \texttt{rtr --system-map}.

The \texttt{router} architecture includes the network topology information
for the router (e.g., blue:black:green)
for Aries and X:Y:Z for gemini). The router information can be populated from
the system-map output or the otuput of \texttt{rtr --interconnect}.
The \texttt{link} architecture represents the network link connectivity,
consisting of type (e.g., Blue, X+) and the tile endpoints obtained
from the interconnect output.

The \texttt{link} architecuture supports determination if an event affects the
components at teh other end of a network link. The router architecutre
supports determination of proximity of events in the network topology.
We chose to separate architecutres in order to enable different
search and interpreation of events which affect components with
different \texttt{association} with each other. \texttt{associations}
include \texttt{parent-child: component-component} as opposed to
\texttt{parent-child: router tile}, or \texttt{peer: HSN link}
as opposed to \texttt{peer: Router-NIC} or \texttt{peer: NIC-Proc}.
To represent more globally associated, such as SMW
events which can directly affect all components, we define
a \texttt{supremum} relationship.

In the prototype, we define all architectures and relationships,
however we currently prinipally search the physical topology only
and support recursive search up and down, including for supremum
relationships. We are working on tools to better enable search of
the multiple architecture represenations.

In order to determine the effect of events on jobs, we also
intend that job data be made availble with the annotations.
We expect only the current common fields (e.g., starttime,
endtime, components, etc) exposed in scheulder logs or
interfaces.







