\section{Case Studies}
\label{s:examples}

We fed Baler with the Cray logs as they resided on the SMW This requires us
to do some file-specific format extraction of messages, timestamps,
and components (e.g., \texttt{netwatch}, \texttt{hwerrlog}) which we may not have had to do if
we fed it raw syslog versions of the files or datastream however, this also enables us
to include the log file type (e.g., \texttt{nlrd}, \texttt{hwerrlog}) in the pattern metadata,
whcih aids the log file look up. For many cases, messages are reported
on the SMW with the component association of the SMW. For some cases, we can
extract the actual component of interest, for example, from the Baler pattern
\texttt{<host> nlrd <pid> found\_critical\_aries\_error: handling failed * link on <host> (node )}
we can infer the fields from which to extract the \texttt{host} and \texttt{node} to which
the annotation should be associated. Other messages refer to actions by the SMW for which
the component cannot be inferred. In these cases the component assignment will either be
the smw or 'unknown'.

We used Baler for all major log processing, except for the \texttt{command} log, which required us
to associate \texttt{START} and \texttt{END} of events, for which we used a perl script. This
log includes both manually initiated and automatically invoked commands of
interest such as warmswaps, boots, etc. This was particularly useful for
determining manual actions that may not have been well documented by the
system administrators. This resulted in another 2000 annotations. In addition,
we extracted the times of reboots from the datatime in the name of the \texttt{p0-XXX}
directories. All annotations from these two sources are attributed as manually induced.

Some other system administrator actions were recorded by manually generated annotations
(about 10, in this case). Ticketing systems may be used to generate such annotations as well.
Knowledge of such events is useful for understanding the root causes and resolution of errors
in the dataset. These annotations were generated manually.

Other non-log events include external actions by non-administrators
such as facilities tests, fault injection research, which require
annotations by differnet people. These were also generated manually
for this dataset. Similar to the Baler Annotations, for the
prototype these were loaded into the same annotation database.

Log data was extracted from alps logs, which was the scheduler in use for
this time period. This could be replaced with queries to a slurm,
or similar, database or interface, where available.

Recall that the main aim of the annotations
is to provide a reduced set of searchable, understandable data, with the
ability to use the annotations to further more detailed search of the raw
logs, if possible and desired. This differs from tools
such as SEC, which is intended to enable action upon the run time
occurence of a log line matching a regex (e.g., notification
of failed component), or Splunk, which is intended
to facilitate knowledge of the occurences of pre-defined
events with accompaning statistical plots.


