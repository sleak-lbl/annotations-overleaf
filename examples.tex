\section{Case Studies}
\label{s:examples}
%%%%%%%%%%%%%%%%%%%%%%%%%

We show use of the annotations using the prototype implementation and tools. Examples
demonstrate the utility of the annotations in understanding
event occurrences and in problem diagnosis. They demonstrate that
the search space is greatly reduced from the whole log files, yet
the key events are revealed. They show how the annotations can be
used to drive further analysis in the log files. They show that
the descriptions can provide contextual information needed
for understanding events, thus lowering the barrier of
expert knowledge needed for understanding log events.

In these examples, we have presented our annotations in their current
state -- including typos and uncertainties in the interpretations.
We expect that this will be reflective of the annotations
in operation, with annotations evolving as additional authors
weigh in and additional expertise is obtained.

Note that some columns in the figures of annotation queries have been suppressed due to space
constraints. We explicitly retain the \texttt{balerpatternid} in the columns as this makes it easier
for the reader to associate individual instances of annotations of the same underlying event type
(e.g., same log message expect for a different component at a different time).
Colors in figures containing annotations are used to help call out annotations
referred to in the text.

\subsection{Job Impact}
\label{s:jobimpact}
%%%%%%%%%%%%%%%%%%%%%%%%%

We have annotated messages relating to potentially performance impacting conditions, including
thermal throttling events, power budgets exceeded, and memory errors. Example annotation
descriptions  include:
\begin{itemize}
\item \scriptsize\texttt{Correctable memory error.  This may result in degraded performance}\normalsize
\item \scriptsize\texttt{Blade or Cabinet controller taking correctable memory errors. This may affect performance.}\normalsize
\item \scriptsize\texttt{Package temperature above threshold (too hot). The CPU clock has been throttled. Should result in all threads for all cores will be throttled. This may affect application performance.}\normalsize
\end{itemize}

Of particular interest in XC systems is the ability to power cap. In such cases, not only is it of interest when
the power budget is exceeded, but also when caps are applied, or perhaps fail to be applied. Our annotation
system enables such events to be exposed to the user.
Many of these messages are identified by commands in the \texttt{commands} file or in the \texttt{controller} logs
and therefore are not typically released to users.
\RED{does the user get warned if setting the power cap fails to be applied?}

Examples include commands annotated as:
\begin{itemize}
\item \scriptsize\texttt{applying a power profile}\normalsize
\item \scriptsize\texttt{enforcing a power limit}\normalsize
\end{itemize}
error messages annotated as:
\begin{itemize}
\item \scriptsize\texttt{Error getting initial node power status. This may affect power capping.}\normalsize
\item \scriptsize\texttt{Error disabling power monitor. This may affect power capping.}\normalsize
\item \scriptsize\texttt{Node Error setting power budget. This may affect power capping.}\normalsize
\end{itemize}

A fundamental goal of the annotations is to enable user to understanding why performance and power limits did not perform as expected.
Basic capabilities enabled by our infrastructure include the ability to determine annotations occurring during a given job
and jobs running while an annotation occurred. Examples of each are given in Figure~\ref{f:powerbudget}.
In the top of the figure, all annotations during a job are queried -- it is seen that multiple times on multiple nodes, the
power budget is exceeded, which may result in performance impact. In the bottom of the figure jobs running while an annotation occurs are queried. In this case,
while the annotation was specific to node \texttt{c0-0c1s3n2}, the job itself was running on 96 nodes. Facilitating
tracing the propagation of impact of an event is one of the design goals of this work.

\begin{figure*}
\begin{annol}

# query for annotations during JobId 163510
python get.py -j 163510  -f table annotations
id	authorid	starttime	endtime		description	logfiles	LDcategory	components	balerpatternid
<@\textbf{\textcolor{red}{855158	acg	2015-05-03 01:12:16	2015-05-03 01:12:16	Node power budget exceeded.	controllermessages	PW	["c0-0c1s3n2"]	2871}}@>
<@\textbf{\textcolor{red}{864565	acg	2015-05-03 01:12:21	2015-05-03 01:12:21	Node now within power budget after it was exceeded.	controllermessages	PW	["c0-0c1s3n2"]	2872}}@>
<@\textbf{\textcolor{red}{855159	acg	2015-05-03 01:12:22	2015-05-03 01:12:22	Node power budget exceeded.	controllermessages	PW	["c0-0c0s11n0"]	2871}}@>
<@\textbf{\textcolor{red}{855160	acg	2015-05-03 01:12:23	2015-05-03 01:12:23	Node power budget exceeded.	controllermessages	PW	["c0-0c1s3n1"]	2871}}@>
<@\textbf{\textcolor{red}{...Occurs multiple times for multiple nodes...}}@>


# query for jobs running during annotation 864574
python get.py --jobs 855158 -f table annotations
JobId    UID    JobName    NumNodes    Start    End
<@\textbf{\textcolor{red}{('163510', XXX, ``'xhpl''', 96, '2015-05-03 00:51:24', '2015-05-03 01:19:15')}}@>>

\end{annol}
\caption{Annotations provide access to power state information in otherwise unavailable logs. Basic implementation
capabilities include discovery of annotations during a job (top) and and jobs running while an annotation occurred (bottom).}
\label{f:powerbudget}
\end{figure*}



\subsection{HSN Congestion}
\label{s:congestion}
%%%%%%%%%%%%%%%%%%%%%%%%%

In order to investigate network conditions, we chose to search for annotations involving the
word ''congest''. Query results are given in Figure~\ref{f:congest}. This looks like the
expected output of response to a congestion conditions, though it was surprising
on a system of this size, with candidate nodes and applications
as computed by the system.

\begin{figure*}
\begin{annol}
# query for annotations where any of the text fields (e.g., description, LDcategory) contain the word 'congest'
python get.py  -t congest -f table annotations
id      authorid   starttime              endtime                description                                                      logfiles    LDcategory    components    balerpatternid
756163    acg      2015-04-28 10:15:44    2015-04-28 10:15:44    System computing and listing congestion candidate applications   nlrd    NE    ["unknown"]    704
756168    acg      2015-04-28 10:15:44    2015-04-28 10:15:44    System computing and listing congestion candidate nodes          nlrd    NE    ["unknown"]    705
...
756167    acg      2015-04-28 10:32:06    2015-04-28 10:32:06    System computing and listing congestion candidate applications   nlrd    NE    ["unknown"]    704
756172    acg      2015-04-28 10:32:06    2015-04-28 10:32:06    System computing and listing congestion candidate nodes          nlrd    NE    ["unknown"]    705
\end{annol}
\caption{Congestion response annotations occur 5 times within 15 minutes. The annotation regarding
candidate applications drove investigation of the \texttt{nlrd} log file, but no applications
were listed.}
\label{f:congest}
\end{figure*}

These annotations guided us to the \texttt{nlrd} in search of applications of interest. 
The link from annotation to relevant slice of the source logfile is a planned capability
of the graph-query tool, however that functionality was not yet available so a visual 
inspection of the relevant file was used.

There were, however, \emph{no} applications listed.
As a result, we then queried for all annotations around this time window to find indication
of a non-application congestion cause.

A query for annotations between 10:00 and 10:32 on that day resulted in 300 annotations, however there are only 7 distinct ones.
The reduction in log lines to annotation instances makes investigation of time ranges tractable and eases discovery
of similar event instances. Other than the ones in Figure~\ref{f:congest}, the rest dealt with problems with a single component \texttt{c0-0c1s8a0n0} and system
response to congestion.
\begin{itemize}
\item  192 occurrences for component \texttt{c0-0c1s8a0n0} of \scriptsize\texttt{Correctable memory error.  This may result in degraded performance.}\normalsize
\item  47 occurrences for component \texttt{c0-0c1s8a0n0} of \scriptsize\texttt{Component failed.}\normalsize
\item \scriptsize\texttt{Telling all blades to throttle network bandwidth. This should result in decreased network injection.}\normalsize
\item \scriptsize\texttt{Telling all blades to unthrottle network bandwidth. This should enable increased network injection.}\normalsize
\item \scriptsize\texttt{Unthrottling the service blades only}\normalsize
\end{itemize}


While we cannot be sure from the annotations alone that failure in this component was the cause of the congestion, it
is clearly a strong suspect.

Querying for annotations for \texttt{c0-0c1s8a0n0} revealed that the component's problems of these 2 types
started on 2015-04-02 10:22:47 and ended on 2015-04-30 at 07:16:52.
Narrowing down the root cause of the problems is difficult, however, because of a number of deliberately induced failures
during facilities testing which occurred that day. While we have put in a manual annotation for \scriptsize\texttt{Facilities testing}\normalsize,
the distributed system which we envision would have enabled the Facilities staff to annotate in more detail the exact testing which occurred.
Currently the \scriptsize\texttt{Facilities testing}\normalsize annotation has to serve as a indicator to examine the logs in which
indications of induced fan and power failures occurred.

The resolution of the problem is discovered by using a depth search to query annotations of related components: here \texttt{-d 2}
includes two levels of parents (\texttt{c0-0c1s8a0} and \texttt{c0-0c1s8}) children (none), and any unknown/supremum components.
The depth was chosen with the expectation that resolution would occur due to actions at the Aries or blade level.

While roughly 500 annotations occurred in response to the query,
only about 30 distinct annotations occurred. This is in contrast to the raw log files in which over 153,000 log lines occurred.
The annotations make it easy to understand the sequence of events. Extracted annotations are in Figure~\ref{f:congestresolve}.
First an annotation of a system administrator, 'abc', action, generically assigned to the day (green) confirms that the blade is being reseated in response
to errors. Warmswaps of the blade occur (cyan), however, while the warmswaps report as successful, timeouts waiting for items in the Outstanding Request Buffer (ORB)
result in the ORB being 'scrubbed', delaying the recovery (red). The annotations help with the understanding of the ORB scrub related events.
Eventually the blade is added back (green) successfully and the blade is then booted.

\begin{figure*}
\begin{annol}

# query for annotations between the time frame of interest for the named component and any components within a depth of 2
python get.py  -c c0-0c1s8a0n0 -d 2 -s "2015-04-30 07:00:00" -e "2015-04-30 10:00:00" -f table annotations

id	authorid	starttime	endtime		endstate    description	logfiles	LDcategory	components	balerpatternid

<@\textbf{\textcolor{green}{4	abc	2015-04-30 00:00:01	2015-04-30 23:59:59	aries errors	blade reseated	Blade reseating in response to aries errors	NE	["c0-0c1s8"]}}@>
<@\textbf{\textcolor{cyan}{865013	acg	2015-04-30 07:10:14	2015-04-30 07:10:15	1	xtwarmswap remove	commands	NO	["c0-0c1s8"]}}@>
<@\textbf{\textcolor{cyan}{865797	acg	2015-04-30 07:15:50	2015-04-30 07:15:50	1	xtwarmswap remove	commands	NO	["c0-0c1s8"]}}@>
<@\textbf{\textcolor{cyan}{866043	acg	2015-04-30 07:16:47	2015-04-30 07:16:47	1	xtwarmswap remove	commands	NO	["c0-0c1s8"]}}@>
<@\textbf{\textcolor{cyan}{865976	acg	2015-04-30 07:16:55	2015-04-30 07:17:25	0	xtwarmswap remove	commands	NO	["unknown"]}}@>
766569	acg	2015-04-30 07:16:56	2015-04-30 07:16:56		Handling Warm swap for partition.	nlrd	NO	["unknown"]	455
866491	acg	2015-04-30 07:16:56	2015-04-30 07:16:56	0	xtcli set_alert		commands	NE	["c0-0c1s8a0"]
752251	acg	2015-04-30 07:17:03	2015-04-30 07:17:03		Setting alerts due to failures. A network reroute is required	nlrd	NE	["unknown"]	498
756229	acg	2015-04-30 07:17:10	2015-04-30 07:17:10		Quiescing the network. This should result in decreased network injection.	nlrd	NE	["unknown"]	509
756239	acg	2015-04-30 07:17:10	2015-04-30 07:17:10		Finished quiescing the network. 	nlrd	NE	["unknown"]	512
865961	acg	2015-04-30 07:17:16	2015-04-30 07:17:23	0	xtcli set_alert		commands	NO	["unknown"]
756132	acg	2015-04-30 07:17:25	2015-04-30 07:17:25		Telling all blades to unthrottle network bandwidth. This should enable increased network injection.	nlrd	NE	["unknown"]	423
756249	acg	2015-04-30 07:17:25	2015-04-30 07:17:25		Unquiescing the network. This will allow normal traffic injection to resume.	nlrd	NE	["unknown"]	554
756259	acg	2015-04-30 07:17:25	2015-04-30 07:17:25		Finished unquiescing the network. This will allow normal traffic injection to resume.	nlrd	NE	["unknown"]	557
<@\textbf{\textcolor{green}{766581	acg	2015-04-30 07:17:25	2015-04-30 07:17:25		Warm swap was successful. This is in response to a operation intended to reset/reinit/replace a component (including network components).	nlrd	NO	["unknown"]	566}}@>
766593	acg	2015-04-30 07:17:25	2015-04-30 07:17:25		The recovery operation for a failed link(s) was successful	nlrd	NE	["unknown"]	563
766603	acg	2015-04-30 07:17:25	2015-04-30 07:17:25		Done handling warm swap. This may not necessarily indicate success (?). This is in response to a operation intended to reset/reinit/replace a component (including network components).		nlrd	NO	["unknown"]	561
757133	acg	2015-04-30 07:17:42	2015-04-30 07:17:42		Starting to quiesce the node (node id might be in nodemask).		controllermessages	NE	["c0-0c1s8"]	2661
758693	acg	2015-04-30 07:17:42	2015-04-30 07:17:42		Finished quiescing the node.		controllermessages	NE	["c0-0c1s8"]	2666
<@\textbf{\textcolor{red}{763170	acg	2015-04-30 07:17:42	2015-04-30 07:17:42		Starting ORB scrub -- removing items in the Outstanding Request Buffer since its been too long for those messages	controllermessages	NE	["c0-0c1s8"]	2660}}@>
764343	acg	2015-04-30 07:17:52	2015-04-30 07:17:52		Finishing ORB scrub -- done removing items in the Outstanding Request Buffer since its been too long for those messages	controllermessages	NE	["c0-0c1s8"]	2669
760262	acg	2015-04-30 07:17:53	2015-04-30 07:17:53		Starting to unquiesce the node.		controllermessages	NE	["c0-0c1s8"]	2670
761831	acg	2015-04-30 07:17:53	2015-04-30 07:17:53		Finished unquiescing the node.		controllermessages	NE	["c0-0c1s8"]	2676
<@\textbf{\textcolor{red}{764926	acg	2015-04-30 07:17:53	2015-04-30 07:17:53		ORB timeout on node (nodes are in the message)		nlrd	NE	["unknown"]	435}}@>
<@\textbf{\textcolor{red}{757134	acg	2015-04-30 07:17:54	2015-04-30 07:17:54		Starting to quiesce the node (node id might be in nodemask).		controllermessages	NE	["c0-0c1s8"]	2661}}@>
758694	acg	2015-04-30 07:17:54	2015-04-30 07:17:54		Finished quiescing the node.		controllermessages	NE	["c0-0c1s8"]	2666
<@\textbf{\textcolor{red}{763171	acg	2015-04-30 07:17:54	2015-04-30 07:17:54		Starting ORB scrub -- removing items in the Outstanding Request Buffer since its been too long for those messages	controllermessages	NE	["c0-0c1s8"]	2660}}@>
764344	acg	2015-04-30 07:18:04	2015-04-30 07:18:04		Finishing ORB scrub -- done removing items in the Outstanding Request Buffer since its been too long for those messages	controllermessages	NE	["c0-0c1s8"]	2669
760263	acg	2015-04-30 07:18:05	2015-04-30 07:18:05		Starting to unquiesce the node.		controllermessages	NE	["c0-0c1s8"]	2670
761832	acg	2015-04-30 07:18:05	2015-04-30 07:18:05		Finished unquiescing the node.		controllermessages	NE	["c0-0c1s8"]	2676
<@\textbf{\textcolor{red}{764927	acg	2015-04-30 07:18:05	2015-04-30 07:18:05		ORB timeout on node (nodes are in the message)		nlrd	NE	["unknown"]	435}}@>

.....

<@\textbf{\textcolor{green}{866031	acg	2015-04-30 07:56:36	2015-04-30 08:03:27	0	xtwarmswap add	1	commands	NO	["c0-0c1s8"]}}@>
766571	acg	2015-04-30 07:56:39	2015-04-30 07:56:39		Handling Warm swap for partition.	nlrd	NO	["unknown"]	455
865720	acg	2015-04-30 07:56:39	2015-04-30 07:56:40	0	xtcli clr_alert	1	commands	NO	["c0-0c1s8"]
865819	acg	2015-04-30 07:56:39	2015-04-30 07:56:39	0	xtcli clr_alert	1	commands	NE	["c0-0c1s8a0"]
866784	acg	2015-04-30 07:56:40	2015-04-30 07:56:40	0	xtcli clr_warn	1	commands	NO	["c0-0c1s8"]
865012	acg	2015-04-30 07:56:46	2015-04-30 08:02:21	0	xtcli clr_warn	1	commands	NO	["c0-0c1s8"]
752253	acg	2015-04-30 08:02:22	2015-04-30 08:02:22		Setting alerts due to failures. A network reroute is required	nlrd	NE	["unknown"]	498
756231	acg	2015-04-30 08:03:08	2015-04-30 08:03:08		Quiescing the network. This should result in decreased network injection.	nlrd	NE	["unknown"]	509
756241	acg	2015-04-30 08:03:08	2015-04-30 08:03:08		Finished quiescing the network. 	nlrd	NE	["unknown"]	512
756251	acg	2015-04-30 08:03:23	2015-04-30 08:03:23		Unquiescing the network. This will allow normal traffic injection to resume.	nlrd	NE	["unknown"]	554
756261	acg	2015-04-30 08:03:23	2015-04-30 08:03:23		Finished unquiescing the network. This will allow normal traffic injection to resume.	nlrd	NE	["unknown"]	557
756134	acg	2015-04-30 08:03:24	2015-04-30 08:03:24		Telling all blades to unthrottle network bandwidth. This should enable increased network injection.	nlrd	NE	["unknown"]	423
<@\textbf{\textcolor{green}{766583	acg	2015-04-30 08:03:24	2015-04-30 08:03:24		Warm swap was successful. This is in response to a operation intended to reset/reinit/replace a component (including network components).	nlrd	NO	["unknown"]	566}}@>
766595	acg	2015-04-30 08:03:24	2015-04-30 08:03:24		The recovery operation for a failed link(s) was successful	nlrd	NE	["unknown"]	563
766605	acg	2015-04-30 08:03:24	2015-04-30 08:03:24		Done handling warm swap. This may not necessarily indicate success (?). This is in response to a operation intended to reset/reinit/replace a component (including network components).		nlrd	NO	["unknown"]	561
<@\textbf{\textcolor{green}{865000	acg	2015-04-30 08:08:34	2015-04-30 08:08:39	0	xtcli boot	1	commands	NO	["c0-0c1s8"]}}@>
\end{annol}
\caption{A blade reseating was performed to resolve blade problems which led to the congestion event.
Multiple iterations of scrubbing the Outstanding Request Buffer (ORB) were needed which
delayed resolution. The annotation of the system administrator (identified by 'abc') action supports the diagnosis.
}
\label{f:congestresolve}
\end{figure*}


The annotations additionally make it easy to compare and investigate timescales of similar events. For instance,
a recovery analysis might be based on the occurrences and durations of the warmswap sequences. This case
might appear of longer duration that others, and the intervening ORB scrubbing events that
needed to be handled for full recovery would be easily apparent.


\subsection{Root Cause Diagnosis}
\label{s:route}
%%%%%%%%%%%%%%%%%%%%%%%%%
Another network investigation started with a search for annotations involving the word ''route''.
We were expecting to see events about reroutes triggered as a result of component failure. More interesting
are cases where the reroute failed. Query output is shown in Figure~\ref{f:routeq}.

\begin{figure*}
\begin{annol}
# query for annotations where any text field contains the word 'route'
python get.py -t route -f table annotations

id	authorid	starttime	endtime	state	description	manual	logfiles LDcategory	components	balerpatternid
752245	acg	2015-02-27 11:53:08	2015-02-27 11:53:08	Setting alerts due to failures. A network reroute is required	nlrd	NE	["unknown"]	498
...
752255	acg	2015-05-08 07:54:15	2015-05-08 07:54:15	Setting alerts due to failures. A network reroute is required	nlrd	NE	["unknown"]	498
752256	acg	2015-05-08 08:11:47	2015-05-08 08:11:47	Setting alerts due to failures. A network reroute is required	nlrd	NE	["unknown"]	498
<@\textbf{\textcolor{red}{756223	acg	2015-05-08 08:17:46	2015-05-08 08:17:46	Error during computation of network route    nlrd	NE	["unknown"]	749}}@>
<@\textbf{\textcolor{red}{756224	acg	2015-05-08 08:31:24	2015-05-08 08:31:24	Error during computation of network route    nlrd	NE	["unknown"]	749}}@>
\end{annol}
\caption{Output of query for route annotations. Complete output = 15 annotations. Occurrences of network reroutes and failures in the rerouting process are of interest.}
\label{f:routeq}
\end{figure*}
%752254	acg	2015-05-08 07:35:42	2015-05-08 07:35:42	Setting alerts due to failures. A network reroute is required	nlrd	NE	["unknown"]	498

Note that the annotation in red \texttt{Setting alerts due to failures. A network reroute is required} makes clear that there has been a failure and what the next step in the response will be. It appears that 3 failures occurred that require network reroutes to recover and two of the computations of those routes failed.

To understand why the route computation failed, we query for annotations during a time frame proceeding the event, limiting the options to network (NE) annotations only.
Query output is shown in Figure~\ref{f:routNEq}. There are only 24 total annotations as opposed to the raw log lines which total over 238,000.
It is clear from the annotations, that the component triggering the problem was \texttt{c0-0c0s9a0} and that
while the recovery operation for a failed link was successful (green), that the
failure of the reroute was due a problem in adding the blade back to the HSN (to include it in the routing) (orange).


\begin{figure*}
\begin{annol}
# query for any annotations within the time range where the LDcategory is 'NE' (network)
python get.py -s ''2015-05-08 08:00:00'' -e ''2015-05-08 08:35:00'' -t LDcat=NE -f table annotations

id	authorid	starttime	endtime	description	manual	logfiles	LDcategory	components	balerpatternid
866450	acg	2015-05-08 08:11:42	2015-05-08 08:11:42	xtcli set_alert	1	commands	NE	["c0-0c0s9a0"]
752256	acg	2015-05-08 08:11:47	2015-05-08 08:11:47	Setting alerts due to failures. A network reroute is required	nlrd	NE	["unknown"]	498
756234	acg	2015-05-08 08:11:49	2015-05-08 08:11:49	Quiescing the network. This should result in decreased network injection.	nlrd	NE	["unknown"]	509
756244	acg	2015-05-08 08:11:50	2015-05-08 08:11:50	Finished quiescing the network. 		nlrd	NE	["unknown"]	512
756254	acg	2015-05-08 08:12:04	2015-05-08 08:12:04	Unquiescing the network. This will allow normal traffic injection to resume.	nlrd	NE	["unknown"]	554
756148	acg	2015-05-08 08:12:05	2015-05-08 08:12:05	Telling all blades to unthrottle network bandwidth. This should enable increased network injection.	nlrd	NE	["unknown"]	423
756264	acg	2015-05-08 08:12:05	2015-05-08 08:12:05	Finished unquiescing the network. This will allow normal traffic injection to resume.	nlrd	NE	["unknown"]	557
<@\textbf{\textcolor{green}{766598	acg	2015-05-08 08:12:05	2015-05-08 08:12:05	The recovery operation for a failed link(s) was successful	nlrd	NE	["unknown"]	563}}@>
864984	acg	2015-05-08 08:16:43	2015-05-08 08:16:43	xtcli clr_alert	1	commands	NE	["c0-0c0s9a0"]
<@\textbf{\textcolor{orange}{752223	acg	2015-05-08 08:17:46	2015-05-08 08:17:46	Marking HSN links down on blades that could not be added	nlrd	NE	["unknown"]	745}}@>
756149	acg	2015-05-08 08:17:46	2015-05-08 08:17:46	Telling all blades to unthrottle network bandwidth. This should enable increased network injection.	nlrd	NE	["unknown"]	423
<@\textbf{\textcolor{red}{756223	acg	2015-05-08 08:17:46	2015-05-08 08:17:46	Error during computation of network route	nlrd	NE	["unknown"]	749}}@>
866808	acg	2015-05-08 08:17:46	2015-05-08 08:17:46	xtcli set_alert	1	commands	NE	["c0-0c0s9a0"]
866328	acg	2015-05-08 08:30:21	2015-05-08 08:30:21	xtcli clr_alert	1	commands	NE	["c0-0c0s9a0"]
<@\textbf{\textcolor{orange}{752224	acg	2015-05-08 08:31:24	2015-05-08 08:31:24	Marking HSN links down on blades that could not be added	nlrd	NE	["unknown"]	745}}@>
756150	acg	2015-05-08 08:31:24	2015-05-08 08:31:24	Telling all blades to unthrottle network bandwidth. This should enable increased network injection.	nlrd	NE	["unknown"]	423
<@\textbf{\textcolor{red}{756224	acg	2015-05-08 08:31:24	2015-05-08 08:31:24	Error during computation of network route	nlrd	NE	["unknown"]	749}}@>
865971	acg	2015-05-08 08:31:24	2015-05-08 08:31:24	xtcli set_alert	1	commands	NE	["c0-0c0s9a0"]
\end{annol}
\caption{Output of query for network related annotations to investigate the cause of the failed routes. Complete output = 24 annotations.
A search of the raw log lines would be much more labor intensive -- 238,000 raw log lines occurred during this period.}
\label{f:routNEq}
\end{figure*}

This, thus drives us to investigate problems with \texttt{c0-0c0s9}. Query output is shown in figure~\ref{f:c0-0c0s9q}. The full output has 90 annotation, but there
are only a few distinct ones (shown).  Messages pertaining to the node (e.g, Error getting power status)(cyan) occur for all nodes on the blade and repeats (multiple nodes and repeats
suppressed). There is an out-of-memory killer annotation (red) that occurs repeatedly (repeats suppressed in the figure). Of particular interest, is that the
out of memory problem is reported by the blade controller (file is controllermessages and component is the blade), as opposed to a user process being killed by
the OOM killer on a node.

\RED{where do we go from here? look at the logs?}

\begin{figure*}
\begin{annol}
# query for annotations within the time range and for the specified component
python get.py -s "2015-05-08 08:00:00" -e "2015-05-08 08:35:00" -c c0-0c0s9 -f table annotations

id	authorid	starttime	endtime	description	logfiles	LDcategory	components	balerpatternid
<@\textbf{\textcolor{red}{864619	acg	2015-05-08 08:06:48	2015-05-08 08:06:48	OOM kill process.		controllermessages	NO	["c0-0c0s9"]	2971}}@>
866418	acg	2015-05-08 08:11:40	2015-05-08 08:12:05	xtwarmswap remove	commands	NO	["c0-0c0s9"]
865065	acg	2015-05-08 08:13:00	2015-05-08 08:13:16	xtcli power down	commands	PW	["c0-0c0s9"]
865709	acg	2015-05-08 08:13:50	2015-05-08 08:15:12	xtcli power up	commands	PW	["c0-0c0s9"]
<@\textbf{\textcolor{red}{864620	acg	2015-05-08 08:13:51     2015-05-08 08:13:51	OOM kill process.		controllermessages	NO	["c0-0c0s9"]	2971}}@>
<@\textbf{\textcolor{red}{...REPEATS 6 Times}}@>
865955	acg	2015-05-08 08:16:41	2015-05-08 08:17:46	xtwarmswap add	commands	NO	["c0-0c0s9"]
864985	acg	2015-05-08 08:16:43	2015-05-08 08:16:43	xtcli clr_alert	commands	NO	["c0-0c0s9"]
865485	acg	2015-05-08 08:16:43	2015-05-08 08:16:43	xtcli clr_warn	commands	NO	["c0-0c0s9"]
865241	acg	2015-05-08 08:19:09	2015-05-08 08:19:09	xtwarmswap remove	commands	NO	["c0-0c0s9"]
865015	acg	2015-05-08 08:19:58	2015-05-08 08:19:58	xtcli halt	commands	NO	["c0-0c0s9"]
<@\textbf{\textcolor{red}{864624	acg	2015-05-08 08:21:01	2015-05-08 08:21:01	OOM kill process.	controllermessages	NO	["c0-0c0s9"]	2971}}@>
<@\textbf{\textcolor{red}{...REPEATS 5 Times}}@>
866482	acg	2015-05-08 08:21:21	2015-05-08 08:21:51	xtcli power down	commands	PW	["c0-0c0s9"]
865041	acg	2015-05-08 08:25:08	2015-05-08 08:26:29	xtcli power up	commands	PW	["c0-0c0s9"]
866012	acg	2015-05-08 08:30:18	2015-05-08 08:31:24	xtwarmswap add	commands	NO	["c0-0c0s9"]
865948	acg	2015-05-08 08:30:21	2015-05-08 08:30:21	xtcli clr_alert	commands	NO	["c0-0c0s9"]
866451	acg	2015-05-08 08:30:21	2015-05-08 08:30:21	xtcli clr_warn	commands	NO	["c0-0c0s9"]
\end{annol}
\caption{Output of query for annotations to investigate the cause of the component failure. Complete output = 90 annotations, about 10 of which are distinct. For example, the
node-related annotations occur for each node on the blade and many repeat in time and are suppressed in the figure. An OOM killer event occurs which is reported by the blade
controller, not a node.}
\label{f:c0-0c0s9q}
\end{figure*}


Finally, we are interested in determining if this problem got resolved and how. We utilize the depth search
\texttt{-d 1} to query parents (\texttt{c0-0c0}), children (the nodes and Aries), and any unknown/supremum components.
The depth and time range currently
are chosen by trial and error, however from output in Figure~\ref{f:routeresolution} it is clear that the OOM messages continue until a unsuccessful attempt is made to power
down the blade (orange), and a few attempts are necessary to rebooted the system (red) and clear the alert (green). This case also illustrates
the endstate field (which is automatically populated with the end state of commands in the command file (described in Section~\ref{s:annotations}))
as well as the manual attribution to any annotations of events from the \texttt{command} file and \texttt{p0-XXX} directories. The latter are
attributed to a generic \texttt{adm} for the annotation, as opposed to the human annotation of the blade reseating in the previous example.

\begin{figure*}
\begin{annol}
# query for annotations between the time frame of interest for the named component and any components within a depth of 1
python get.py -s "2015-05-08 08:35:00" -e "2015-05-08 23:35:00" -c c0-0c0s9 -d 1 -f table annotations

id	authorid	starttime	endtime		endstate	description	manual	logfiles	LDcategory	components	balerpatternid
<@\textbf{\textcolor{orange}{865860	acg	2015-05-08 08:40:24	2015-05-08 08:40:54	0	xtcli power down	1	commands	PW	["c0-0c0s9"]}}@>
866403	acg	2015-05-08 08:51:18	2015-05-08 08:52:39	0	xtcli power up	1	commands	PW	["c0-0c0s9"]
865969	acg	2015-05-08 09:04:56	2015-05-08 09:05:09	0	xtcli power up	1	commands	PW	["c0-0c0s9"]
865426	acg	2015-05-08 10:55:04	2015-05-08 10:55:21	0	xtcli shutdown	1	commands	NO	["unknown"]
865427	acg	2015-05-08 10:59:07	2015-05-08 10:59:08	0	xtcli clr_alert	1	commands	NO	["c0-0c0s9"]
865429	acg	2015-05-08 10:59:07	2015-05-08 10:59:08	0	xtcli clr_alert	1	commands	NE	["c0-0c0s9a0"]
866614	acg	2015-05-08 10:59:08	2015-05-08 11:00:13	0	xtcli halt	1	commands	NO	["unknown"]
865078	acg	2015-05-08 11:00:29	2015-05-08 11:00:29	1	xtcli power on	1	commands	PW	["c0-0c0s9"]
866365	acg	2015-05-08 11:00:42	2015-05-08 11:00:53	0	xtcli power up	1	commands	PW	["c0-0c0s9"]
<@\textbf{\textcolor{orange}{766617	acg	2015-05-08 11:04:02	2015-05-08 11:04:02		Boot manager - halt request has failed		bm	NO	["unknown"]	15732}}@>
866399	acg	2015-05-08 11:04:02	2015-05-08 11:04:02	0	xtcli halt	1	commands	NO	["c0-0c0s9"]
865284	acg	2015-05-08 11:05:57	2015-05-08 11:05:57	1	xtcli power	1	commands	PW	["unknown"]
<@\textbf{\textcolor{red}{74	adm	2015-05-08 11:15:31		reboot (p0)	1			NO	["unknown"]}}@>
866035	acg	2015-05-08 11:15:42	2015-05-08 11:22:59	1	xtcli power up	1	commands	NO	["unknown"]
866268	acg	2015-05-08 11:26:54	2015-05-08 11:26:54	0	xtcli slot_off	1	commands	NO	["c0-0c0s9"]
864961	acg	2015-05-08 11:27:06	2015-05-08 11:27:06	1	xtcli power slot_off	1	commands	PW	["c0-0c0s9"]
866436	acg	2015-05-08 11:27:18	2015-05-08 11:27:48	0	xtcli power down_slot	1	commands	PW	["c0-0c0s9"]
<@\textbf{\textcolor{red}{75	adm	2015-05-08 11:55:20		reboot (p0)	1		NO	["unknown"]}}@>
866564	acg	2015-05-08 11:55:35	2015-05-08 11:57:04	1	xtcli power down_slot	1	commands	NO	["unknown"]
866547	acg	2015-05-08 12:42:12	2015-05-08 12:42:12	1	xtcli power slot_off	1	commands	PW	["c0-0c0s9"]
866460	acg	2015-05-08 12:42:21	2015-05-08 12:42:52	0	xtcli power down_slot	1	commands	PW	["c0-0c0s9"]
865783	acg	2015-05-08 12:43:11	2015-05-08 12:43:11	0	xtcli disable	1	commands	NO	["c0-0c0s9"]
<@\textbf{\textcolor{red}{76	adm	2015-05-08 12:43:43		reboot (p0)	1			NO	["unknown"]}}@>
864945	acg	2015-05-08 12:43:54	2015-05-08 12:50:15	0	xtcli disable	1	commands	NO	["unknown"]
<@\textbf{\textcolor{green}{866412	acg	2015-05-08 12:50:18	2015-05-08 12:50:18	0	xtcli clr\_alert	1	commands	NE	["c0-0c0s9a0"]}}@>
\end{annol}
\caption{Output of query for annotations to investigate the resolution of the component failure. Attempts to address the blade itself were unsuccessful,
and several reboots were required before the alert cleared.}
\label{f:routeresolution}
\end{figure*}

\RED{Are the endstate codes rationale for this case?}



