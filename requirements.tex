\section{Requirements for a Solution}
\label{s:requirements}

%(from above: In summary the diversity and volume of data, distribution of expertise, risks 
%around publication and challenges of discovery have limited our ability to 
%extract useful insights from the data we collect. In the next section we explore 
%the requirements for a solution to address these constraints.)

Much data is collected - or can be collected - but often by different
people and for a specific purpose, and in formats chosen to suit the 
data source or a specific use for the collection. 
For example log data frequently originates 
as messages emitted by some software and is thus stored as
text files with a timestamped line per entry, while job records
are typically stored in database tables with well-defined fields.
Making data presentable and accessible beyond the specific needs of those
collecting it is difficult and time-consuming and, consequently, 
infrequently performed. 

This implies that to support extration of useful insights from available data 
we need to be agnostic towards the format and storage of the data
\textbf{(requirement 1)}.

Furthermore, staff are often unaware of the full suite of data collection 
activities at their site, let alone farther afield. This leads to missed 
opportunities for gaining insights from what is already collected and 
to redundant collections, so \textbf{requirement 2} is to support 
discovery with no a priori knowledge of other collection efforts.

An effective solution within a site is 
use of a tool like LogStash~\textcolor{red}{cite} 
to convert everything to a common format and collect it in a single 
centralized location. This however has some limitations:

\begin{itemize}
\item It requires all collection activities to ensure their data can
      be converted and stored in the centralized format. Many ad-hoc
      colelction activities cannot justify the effort required to 
      comply, so the centralized collection fails to capture the full
      suite of available data.
\item The diversity of security domains poses a challenge: should data
      be captured to a higher-security domain, filtered to a lower-security
      domain or should each security domain have its own storage?
\item A centralized solution requires a significant commitment to
      centralized maintenance, and trust that the incoming data is 
      in a useful format, at a useful cadence and
      tractable volume?)
\end{itemize}

Therefore we would like our solution to be decentralized 
\textbf{(requirement 3)}.

Access to log datasets is an obvious requirement and the release 
of some such datasets is a goal of the HMDR project\textcolor{red}{CITE}. 
But it is well-known in the research 
community that very few datasets are released for researchers
due to the presence of sensitive data such as user login information, 
the difficulty of log anonymization and the limited cost-benefit 
trade-off between helping the community and the risk of mistakenly 
releasing sensitive information. 

The effort and risk in a release paradigm of ``carefully redact 
sensitive information before releasing data'' is a roadblock for 
publishing datasets so a solution should promote the opposite 
paradigm of ``select some non-sensitive data and release that'' 
instead \textbf{(requirement 4)}.

Expert commentary on the meaning of log messages has been identified as
a desirable resource ~\cite{CUG2016BoF} but expertise is domain-specific
and distributed across many individuals, all of whom have other, primary 
responsibilities. The solution must therefore allow for
domain experts to independently contribute advice in the format they
already use \textbf{(requirement 5)} - for example a
spreadsheet of log message definitions or a ticketing system with
maintenance requests and notes.

Failure analysis research often looks for relationships between 
and sequences in log records, and might include comparisons against 
``control'' logs from a different period or system. Operational 
troubleshooting seeks to identify possibly-contributing events in the lead 
up to an identified failure. 

Failures at one component may be triggered by events 
at a connected component so an ideal search would extend to logs from
related components. These relationships may not be a simple hierarchy
 - for example the failure of a link might affect two ``peer'' nodes.
But an overly-wide search will return an intractable volume of data, so
the solution should support some means of filtering data as well as 
of discovering non-obviously-related data \textbf{(requirement 6)}.

% key para:
So in summary, our solution should:
\begin{enumerate}
\item Be agnostic towards the format and storage of data.
\item Not require a priori knowledge of other similar efforts.
\item Be decentralized.
\item Allow publishing of data to be low effort and low risk.
\item Allow domain experts to independently contribute advice 
      in a format convenient for them.
\item Support data discovery and filtering across related components
\end{enumerate}

